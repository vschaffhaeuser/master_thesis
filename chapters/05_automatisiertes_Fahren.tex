% !TeX root = ../main.tex
% Add the above to each chapter to make compiling the PDF easier in some editors.

\chapter{Abgleich der Sicherheitsrelevanz von Fahrsituationen für menschliche Fahrer mit denen für automatisierte Fahrzeuge}\label{chapter:automatisiertes Fahren}

"Die Motivation zur Teilnahme am Straßenverkehr ist Mobilität, nicht Sicherheit." \parencite[S.146]{Huguenin.2017}

%Es werden deshalb die ganzen FAS erläutert, da eine Zusammenführung bestehender Assistenzsystem zum automatisierten und vernetzten Fahren das wahrscheinlich größte potenzial hat.

%\Textcite[S.3]{Gasser.2011} gibt einen tabellarischen Überblick über die Klassifizierung automatisierter FAhrfunktionen nach bast. Daber wird in Driver Only Assistiert, Teilautomatisiert, Hochautomatisiert und Vollautomatisiert unterschieden.

"Autonomes Fahren: Die Fahraufgabe nach Donges wird \enquote{vollautomatisiert} ausgeführt. Diese Definition wird erweitert um die Annahme, dass die Ausführung der Fahraufgabe auf Basis maschinell autonomen Verhaltens, innerhalb einer vorher festgelegten Verhaltensrahmens, geschieht." \parencite[S.34]{Maurer.2015}

Einige wenige Unfallkonstellationen werden sich auch durch die Erhöhung des Automatisierungsgrades nicht verhindern lassen. Zusätzlich muss bei höherer Automatisierung ein spezifisches Automatisierungsrisiko berücksichtigt werden. ("Dieses Risiko wird begrenzt durch Maßnahmen der funktionalen Sicherheit in Bezug auf elektrische und elektronische Fehler sowie die bei Nutzung von Sensorik zwangsläufig begrenzte Wahrnehmung von Vorgängen in der Außenwelt.") \parencite[S.5]{Gasser.2011}
%Gasser stellt zudem ein Diagramm zur Wirkung der Fahrzeugautomatisierung das. Übernehmen, da darin die erhoffte Wirkung gar nicht so groß ist.

"Der Fahrer ist bei der Entstehung von Unfallsituationen häufig nicht in der Lage, ein vollständiges Situationsmodell aufzubauen und kann somit nicht angemessen auf die Situation in der Umgebung reagieren." \parencite[S.48]{Zademach.24.09.2015} Der Aufbau eines Situationsmodells wird durch sich potenziell bewegende Objekte erschwert. Dies liegt vor allem darin, dass der Fahrer das Verhalten dynamischer Objekte nur schwer vorhersagen kann und sich dadurch die Wahrscheinlichkeit unvorhergesehener Ereignisse erhöht.

FAS bestehen aus drei Grundkomponenten:

\begin{itemize}
	\item Sensorik (Zuständig für die Informationserkennung)
	\item Hard- und Software (Verarbeitet die durch die Sensoren erfassten Informationen und gleicht diese mit dem Soll-Zustand ab)
	\item Aktorik oder auch Mensch-Maschine Schnittstelle genannt (Gleicht den Unterschied zw. dem Ist- und Soll-Zustand aus) \parencite[S.10]{Blakaj.14.09.2016} 
\end{itemize}

%Auf den folgenden Seiten wird ein Überblick über vorhandene FAS gegeben. Welche Ansaätze könnten für das automatisierte Fahren im urbanen Raum hilfreich sein? Schaubild zur Wirkung der Fahrzeugautomatisierung von Gasser selbst übernehmen.
Beim voll automatisierten Fahren würde die dritte Komponente wegfallen, da der Fahrer nicht mehr verpflichtet ist einzugreifen bzw. bei Fahrten zum Parkplatz gar kein Fahrer an Bord ist.

Drei Stufen von FAS:
\begin{itemize}
	\item Systemen, die den Bewegungszustand des Fahrzeugs erkennen und mit dem Fahrerwunsch vergleichen können.
	\item Systeme, die zusätzlich durch fahrzeugseitige Sensoren bereitgestellt Informationen über das Umfeld verwenden. (Automatische Schutzmechanismen)
	\item Stufe drei wird erreicht, wenn dem Fahrzeug zusätzliche durch die eigene Sensorik nicht zugängliche Informationen zur Verfügung gestellt werden. (C2C, C2I, C2X) 
\end{itemize}
Die drei Stufen werden bei der Entwicklung nicht nacheinander durchlaufen, sondern überschneiden sich \parencite[S.88]{WissenschaflicherBeiratbeimBundesministerfurVerkehrBauundStadtentwicklung.2011}.

Assistenzsysteme können in vier Gruppen unterteilt werden: Informationssysteme, Warnsysteme, aktiv unterstützende Systeme und eingreifende Systeme. Eine Erweiterung um eine fünfte Gruppe würden dann Systeme darstellen, die selbständig bestimmte Aktionen übernehmen, ohne dass der Fahrer dies verhindern oder korrigieren kann \parencite[S.13]{Vollrath.2006}.

"Es sollte daher geprüft werden, ob für autonom arbeitenden, nicht übersteuerbare Systeme mit erheblichem unfallvermeidendem oder schützendem Potenzial eine besondere Form der Typgenehmigung möglich ist, die die Risiken der Erprobung unter realen Bedingungen für die Herstelle, Halter und Versicherungen kalkulierbar macht" \parencite[S.89]{WissenschaflicherBeiratbeimBundesministerfurVerkehrBauundStadtentwicklung.2011}. 

%Klinger nennet Beispiele von FAS, die aktiv in die Fahrzeugsteuerung eingreifen.
"Die Fehlerrate eines konventionellen Fahrers wird mit ca. 2,2*10hoch-4 Unfallbeteiligungen/Stunde geschätzt. Die Anforderungen an autonome Fahrzeuge liegen mit 10hoch-9 Unfallbeteiligungen/Stunde extrem hoch" aktuell ist die Umfelderkennung weit davon entfernt solch einen Wert zu garantieren. Daraus ergibt sich, dass die Voraussetzung für jegliche Art autonomen Fahrens zunächst die sensorielle Wahrnehmung der Umgebung darstellt. Vor allem im urbanen Bereich sind für den Nachweis der Sicherheit außerordentlich umfangreiche Feldtests notwendig \parencite[S.246-249]{Klingner.2017}.

Empfehlungen für FAS bei gefährlichen Abbiegesituationen mit Radfahrern:
"Erforderlich ist die Detektion von Radfahrern, die sich mit vergleichsweise hohen Geschwindigkeiten bei oft gleichzeitig eher geringer Abbiegegeschwindigkeit des Kfz von hinten nähern."
"Ebenso sollten Assistenzsysteme auch links fahrende Radfahrer erkennen können und auch beim Linksabbiegen unterstützen." \parencite[S.310]{Schreiber.2014b}

\Textcite[S.55]{Bremer.2004} weißt drauf hin, dass der Schilderwald in den Städten immer dichter wird. Es ist selbst für den menschlichen Fahrer schwer das ihn betreffende Schild rechtzeitig zu erkennen. Häufig führen zu viele sich zum Teil widersprechende Schilder dazu, dass der Fahrer überfordert ist. Kann dieses Problem durch automatisiertes Fahren gelöst werden?

Bei der Entwicklung automatisierter Fahrzeuge darf die Planerkennung nicht vernachlässigt werden. Was können Sensoren wirklich alles erfassen? Ist es möglich Absichten anderer Verkehrsteilnehmer, vor allem von Fußgängern und Radfahrern, zu erkennen? Die Möglichkeit Hypothesen über die Absichten von Verkehrsteilnehmern zu erstellen, ist die Voraussetzung für die Erkennung kritischer Situationen \parencite[S.30]{MockHecker.1994}. %Auf S.33 wird auch noch auf die Konflikterkennung eingegangen, evtl. beides mit aufnehmen. S.41-42 Wie kann man mit unsicherem Wissen über den Planablauf umgehen? S.5 Plankonflikterkennung, Unterschied Konflikte in der Verkehrswelt zu Konflikten in allg. nichtlinearen Plänen.

\Textcite[S.639]{Kossak.2017} nennt folgende  zwingende Bedingungen für eine vertretbare volle Automatisierung im Straßenverkehr: "alle Automobile:
\begin{itemize}
	\item können sämtlich Hindernisse in der unmittelbaren Umgebung rechtzeitig erkenn und korrekt identifizieren
	\item verfügen über jederzeit perfekt aktualisierte  Straßenkarten und
	\item sind mit einder Software ausgestattet, die absolut einwandfrei funktioniert."
\end{itemize}
%Kossak zitiert selbst, original Quelle nennen?!
Er weist zudem auf die Anfälligkeit digitaler Systemen in bestimmten Bereichen hin. "Schnee, Hagel, Starkregen oder Vereisung der Monitore und Sensoren können Fehlerfunktionen bewirken.Die Spiegelung der Sonne in Fenstern der Straßenbebauung hat bereits zu kritischen Systemstörungen selbst im konventionellen Bereich der FAS geführt."
Ein weiteres Problem könnte sich dadurch ergeben, dass AVs wahrscheinlich so programmiert werden, dass es zu gut wie Möglich vermieden wird Fußgänger zu treffen und zu verletzten. Die könnte dazu führen, dass Fußgänger und Radfahrer die Situation ausnutzen und die Autos "ärgern".
Zudem "erschwert es die Vielfalt unterschiedlicher und/oder unterschiedlich positionierter Verkehrszeichen insbesondere in Sonderzonen (z.B. Baustellen) den automatischen Systemen, Situationen korrekt zu erkennen und einzuordnen." Sind die Schilder schlecht positioniert oder wegen mangelhafter Wartung schlecht erkennbar kann es zum Teil dazu führen, dass es unmöglich ist sie zu erkennen. Dies gilt nicht nur für Schilder, sonder auch für Worte und Texte die oft Zusammen mit Verkehrsschildern als Ergänzung angebracht werden. Hier kommt noch dazu, dass sie in anderen Ländern in einer "fremden" Sprache auftreten.

\Textcite[S.18-21]{Gschwendtner.2015} geht auf vorhanden FAS im Bereich niedriger Geschwindigkeiten ein um Sachschadensunfälle zu vermeiden. Es gibt Bereits Systeme die in diesem Bereich wirken (z.B. Einpark-Assistent, Einparkhilfe). Dies werden aktuell häufig als Komfortsysteme vermarktet, da sie nur einen indirekten Einfluss haben. Es gibt erst wenige Systeme die nicht nur warnen, sondern auch direkt in Eingreifen. Auffällig ist zudem, dass bei allen vorhandenen FAS die Fahrzeugflanken, bei denen es bei einer Beschädigung zu hohen Reparaturkosten kommt, vernachlässigt werden. Dies soll in Zukunft durch eine Rundumüberwachung vermieden werden. Zudem erhofft man sich zukünftig z.B. von Valet-Parking-Systemen oder FAS-Systemen die vom Nutzer gesteuert werden (z.B. Handysteuerung) großes Potential im Bereich der Sachschadensunfälle.
%In einer Tabelle auf S.52 wird zusätzlich die Relevanz von Fahrzeugassistenzsystemen in vierschiednen Bereichen dargestellt. Es wird die Automatische Notbremse, Einparkhilfe, Kreuzungsassistent, Spurwechselassistent und Spurverlassenswarnung dargestellt. Evtl. die Tabelle Kopieren. Ich find die ganz gut.

\Textcite[S.29]{Gerstenberger.17.02.2015} stellt fest, \enquote{dass für die Unterstützung des Fahrers durch Assistenzsysteme am Knotenpunkt vor allem die Einschätzung des kritischen Verhaltens notwendig ist.} Dabei müssen vor allem Sichtbehinderungen bei der Annäherung und die Fahrabsicht anderer Verkehrsteilnehmer erkannt werden.

"Die durch die Anwendung von Kommunikation (V2V, V2I) ermöglichte räumliche und zeitliche Erweiterung des Wahrnehmungshorizonts erlaubt es, potentielle Gefahrensituationen im Voraus zu erkennen und mögliche Verkehrsunfälle zu vermeiden." \parencite[S.59]{Gerstenberger.17.02.2015}

Gerstenberger nennt auf S.160 mögliche Ansätze zur Adressierung grundlegender Ursachen von Knotenpunktunfällen durch Maßnahmen im System Mensch-Straße-Fahrzeug.

Systeme die dazu dienen die Fahrbahn zu beobachten sind z.B. Videokameras, Radar- oder Lasersensoren können für die Abstands- und Geschwindigkeitserkennung verwendet werden \parencite[S.4-5]{Schmidt.2010}.

Viele Unfälle entstehen durch Abstandsfehler, hier ist eine gute Unterstützung durch FAS möglich \parencite[S.25]{Schmidt.2010}.

%Auf S.33 der Arbeit von Schmidt wird die Funktion des Stauassistenten beschrieben.

%Gründel stellt in seiner Arbeit verschiedene FAS vorr. Darunter befinden sich unter anderem: die Automatische Notbremse, ACC, Spurassistenzsystem, Verkehrszeichenerkennung, Fußgängererkennung und Spurwechselassistent. Bei der Erstellung des Kapitels nochmal durchgeheh, welche Ansätze können beim automatisierten Fahren im urbanen Raum übernommen werden? Was hilft dabei Konfliktsituationen zu vermeiden oder Unfälle zu verhindern? Es muss im Hinterkopf behalten werden, dass die Arbeit von 2005 ist. Es gibt neuere Systeme. Die Ansätze können aber trotzdem weiterhelfen.

"Spezifische Wirkungen sind unmittelbare Auswirkungen des Agierens des Systemes sowie die Reaktionen des Fahrers auf Informationen, Warnungen und Eingriffe des Systems. Unspezifische Wirkungen sind die mittelbaren Effekte, die sich aus dem Vorhandensein eines Fahrerassistenzsystems für die Fahrweise ergeben." \parencite[S.50-51]{Grundl.2005}

%Gründl stellt auf S.224 FAS zur Erkennung von Verkehrszeichen vor z.B. Ampeln, Stoppschild...
"Die Erkennung des Ampellichts könnte aus technischer Sicht ein Problem darstellen, Sensoren die dies zuverlässig ermöglichen sind noch sehr kostspielig." \parencite[S.230]{Grundl.2005}%stand 2005 hat sich was geändert?
%S.239 Empfehlungen für die Gestaltung eines Spurwechselassistenten, S.248 Anforderungen. Toterwinkel nicht entscheidend, sonder die Erkennung von Fahrzeugen die sich mit hoher Geschwindigkeit näher. Im urbanen raum nicht relevant.
"Das vielversprechendste FAS stellt die Automatische Notbremse dar." \parencite[S.243]{Grundl.2005}
"Besonders wünschenswert wäre es, wenn nicht nur der vor dem Fahrzeug liegende Bereich sonder ebenso die seitlichen Bereiche des Fahrzeugs erfasst würden, sp dass vpn der Seite nahende Gefahren erkannt werden können." \parencite[S.244]{Grundl.2005}

%Vollrath stellt in seiner Arbeit in dem Kapitel 2.4 Assistenzsysteme vor, die sich schon auf dem Makrt befinden oder noch in der Entwicklung sind. 
%Meitinger gibt in seiner Arbeit in Kapitel 1.2 einen Überblick über Aktive Sicherheitssysteme und Farerassistenzsysteme

Anforderung an aktive Sicherheitssysteme stellen sowohl Kunden, die Fahrzeughersteller als auch die öffentliche Hand. Diese Anforderungen können sich zum Teil wiedersprechen \parencite[S.8]{Meitinger.2008}. %Alle Anforderungen werden in einem Bild dargestellt.

"Die Funktion des mechatronischen Systems ist von den Schnittstellen zwischen logischer und physikalischer Ebene gekennzeichnet, den Sensoren und Aktoren." \parencite[S.9]{Meitinger.2008} Ebenso spielen die Möglichkeiten und die Qualität der Erfassung der Fahrumgebung eine wichtige Rolle.

"Ziel eines Aktiven Sicherheitssystems ist die Verbesserung der Verkehrssicherheit. Wie groß der Sicherheitsgewinn ist hängt von der Anzahl der relevanten Situationen, die in einem betrachteten Zeitraum auftreten, der Wirksamkeit des Systems, seiner konstruktiven Sicherheit und dem Schutz gegen Missbrauch ab." "Es besteht die Gefahr, dass die Sicherheit durch ein Sicherheitssystem nicht verbessert, sonder verschlechtert wird. Grund dafür ist die Risikohomöstase." Sie führt dazu, dass sich der Fahrer durch das System sicherer fühlt und aufgrund dessen ein höheres Risiko eingeht. \parencite[S.10]{Meitinger.2008}

Es kann zwischen den vier Handlungen - Information, Warnung, Unterstützung und autonomer Eingriff - von FAS unterschieden werden. Hierbei ist die Wirksamkeit von autonom eingreifenden Systemen an höchsten einzuschätzen \parencite[S.11]{Meitinger.2008}.

Aktive Sicherheitssysteme sind effizienter, wenn sie gezielt gegen Unfalltypen die  häufig auftreten wirken. Hierfür ist die absolute Anzahl der Unfälle von Bedeutung \parencite[S.19]{Meitinger.2008}. "Die Daten, die für die Abschätzung der Wirksamkeit von Aktiven Sicherheitssystemen nützlich sein können, liegen oft nur spärlich von, sind nicht ausgewertet oder fehlen." \parencite[S.22]{Meitinger.2008}

%\Textcite[S.2]{Schendzielorz.21.09.2016} will mit Hilfe von V2I und V2V Kommunikation und den entsprechenden Infrstrukturseitigen Einrichtungen ein System entwickeln, dass prüft, ob der Fahrer vor einer Konfliktsituation gewarnt werden muss. Hierfür sollen die Knotenpunkte aus der Vogelperspektive betrachtet werden, dafür werden Laserscanner auf Ampel- oder Lichtmasten montiert. Es werden die Szenarien Rotlichtverstoß, Links- und Rechtsabbiegen betrachtet.
Auf S.22 erwähnt er zwei Hauptkategorien von Kreuzungsassistenten \enquote{stand-alone} und \enquote{cooperative}. Während bei den stand-alone Systemen nur die Infrastruktur oder das Fahrzeug dafür verantwortlich ist die Situation richtig einzuordnen treffen Kooperativesystem ihre Entscheidungen auf ausgetauschten Daten (z.B. V2V, V2I).
Urbane Fahrsituationen werden im Projekt UR:BAN in deutschen Städten betrachtet und analysiert. Ziel ist die Entwicklung von FAS und Verkehrsmanagementsystemen. http://urban-online.org/de/urban.html

"Um die Realisierung und Markteinführung von Fahrzeugfunktionen wie autonomes Fahren, im Sinne eines selbständigen Zurechtfindens des Fahrzeugs im Straßenverkehr, überhaupt in greifbare Nähe zu rücken, werden große Anforderungen an die Sicherheit, Zuverlässigkeit und Verfügbarkeit der Software sowie der erfüllenden Systeme gestellt." \parencite[S.3]{Hillenbrand.2012} %Die Softwareumfänge verdoppeln sich mit jeder neuen Fahrzeuggeneration.

Aus der Unfallstatistik ist zu erkenne, dass Fahrerassistenzfunktionen und sicherheitsrelevante Funktionen in Fahrzeugen einen Beitrag zur Sicherheit im Straßenverkehr leisten. Ein Beispiel ist das von Bosch entwickelte elektronische Antiblockiersystem (ABS) oder das Elektronische Stabilitätsprogramm von Mercedes \parencite[S.4]{Hillenbrand.2012}.

"Durch die Unterlassung oder die nicht korrekte Ausführung des Systemfunktionen von sicherheitsbezogenen Systemen können Gefährdungen für die Umgebung, des Systems entstehen. Für die Systementwicklung ist es notwendig, diese Gefährdungen früh zu erkennen um das System im Folgenden so zu entwickeln, dass das mit der Verwendung oder dem Einsatz des Systems in Verbindung stehende Risiko gering gehalten wird." \parencite[S.49]{Hillenbrand.2012}
"Gibt es ernstzunehmende Gefährdungen, so muss das System dahingehend zu entwickeln, dass diese möglichst vermieden werden, bzw. dass die Wahrscheinlichkeit ihres Auftretens dem Grad der Gefährdung sowie dem damit verbundenen Risiko angemessen ist. Diese Angemessenheit wird durch Normen festgelegt." \parencite[S.50]{Hillenbrand.2012} %Satzbau nach originalem Zitat. Bei Verwendung nicht direkt zitieren, sonder umschreiben.

"Ein enormes Potenzial, Verkehrsunfälle deutlich zu reduzieren bietet das automatisierte Fahren. Über 90 Prozent  aller Unfälle sind heute auf menschliches Fehlverhalten zurückzuführen. Mit dem Einzug von Fahrcomputern werden wir die Fahrer deutlich entlasten und kritische Verkehrssituationen massiv reduzieren. Der Sprung zum automatisierten und vernetzten Fahren ist damit nicht nur die größte Mobilitätsrevolution seit der Erfindung des Automobils, sondern bringt auch ein großes Plus an Sicherheit." \parencite[S.4]{DEKRA.2017}

Wenn es gelingt rechtliche und technische Hürden zu überwinden könnte das automatisierte Fahren ein wesentlicher Schlüssel für die längerfristige Entwicklung hin zur Annäherung an das Ziel der "Vision Zero" sein \parencite[S.16]{DEKRA.2017}.

"Die Wirkung elektronischer Fahrerassistenzsysteme kann sich nur entfalten, wenn diese über das ganze Fahrzeugleben zuverlässig funktionieren. Der periodischen Fahrzeugüberwachung kommt dabei noch mehr Bedeutung zu." \parencite[S.62]{DEKRA.2017}

Der Mensch wir häufig als schwächstes Glied im System Straßenverkehr angesehen. Diese Betrachtung führt zu Annahme, dass die Technik aufgrund ihrer geringeren Fehleranfälligkeit dem Menschen überlegen ist. Daraus ergibt sich der Schluss, dass die menschlichen Schwächen mit Hilfe der Technik beseitigt werden sollen \parencite[S.270]{Burg.2017}.

Der Vorteil von Kommunikationslösungen besteht darin, dass die Anforderungen an die \enquote{konventionelle} Umfeldsensorik sinken. Zudem sind Informationen über den Querverkehr trotz eventuellen Sichtbehinderungen verfügbar. Nachteilig ist hingegen, dass alle Fahrzeuge eine gewisse Mindestausrüstung enthalten müssen um den Anforderungen gerecht zu werden \parencite[S.2]{Mages.2008}.

Um Unfälle mit Hilfe einer Kreuzungsassistenz vermeiden zu können müssen vielfältige Anforderungen erfüllt werden. Dazu zählt die Erfassung von Kreuzungen und Vorfahrtsregelungen, das Erkennen des nächsten Phasenwechsels von LSA, die Berücksichtigung vorausfahrender Fahrzeuge und Fußgänger sowie das Abschätzen der Gefahr von Kollisionen mit dem Querverkehr \parencite[S.9]{Mages.2008}.

"Das Unfallvermeidungspotential aktiver Sicherheitssysteme steht und fällt mir der Anzahl der ausgerüsteten Fahrzeuge." \parencite[S.9]{Mages.2008}

Es gibt verschiedene Möglichkeiten den Fahrer anhand von Assistenzsystemen zu unterstützen. Infrastructur-Only(IO)-Systeme sind Systeme die an markanten Punkten z.B. an einer Kreuzung installiert werden. Die Systeme sind nicht ans Fahrzeug gebunden und daher auch nicht auf Neuerungen im  Fahrzeug angewiesen. Häufig werden sie in Form von aktiven Verkehrszeichen ausgeführt. Fahrzeugautarke (FAT) Systeme stellen Systeme dar, deren Funktion weder auf Komponenten in der Kreuzung noch auf Systeme in anderen Fahrzeugen angewiesen ist. Zur Umfelderfassung werden Sensoren im Fahrzeug benötigt, die Informationen über Vorfahrtsregelungen werden aus GPS mit digitalen Karten oder Kamerasystemen gewonnen. Die Fahrzeug-Fahrzeugkommunikation (C2C) funktioniert nur zwischen ausgestatteten Fahrzeugen, kann dafür z.B. bei eingeschränkten Sichtbereichen an Kreuzungen besser agieren. Kooperative Systeme können Systemkomponenten der Infrastruktur mit Elementen im Fahrzeug kombinieren. Das Fahrzeug kann so beispielsweise Informationen zum Phasenwechsel einer LSA erhalten \parencite[S.23-26]{Mages.2008}.

Hohe Zahlen an Unfällen werden fälschlicherweise dem Menschen zugeordnet. Das System kann auch mangelhaft sein \parencite[S.147]{Huguenin.2017}. %Es gibt auch eine Tabelle, in der die Stärken und Schwächen von Mensch und Maschine gegenübergestellt werden.

%Auch interessant bei automatisierten Fahrzeugen, können sie LSA/Schilder bei tiefstehender Sonne erkennen?

\subsection{Rechtlichehintergründe}%noch anpassen

Bei Hoch- bzw. Vollautomatisierung würde der Fahrer, in den Phasen die autonom gesteuert werden, gegen seine Pflichten aus der StVO verstoßen. Hier heißt es, de Fahrer muss jederzeit in das Verkehrsgeschehen eingreifen können. Ziel der Vollautomatisierung ist jedoch, dass der Fahrer sich während der Fahrt anderen Aufgaben widmen kann und nicht permanent den Fahrtverlauf überwachen muss oder Fahrzeuge Strecken sogar ohne Fahrer zurücklegen. Die Nutzung von Hoch- bzw. Vollautomatisierung ist somit nicht mehr als zulässig einzustufen, weil sie eine Abwendung des Fahrers von seiner Fahraufgabe vorsehen. Die Haftung des Fahrzeughaltes bleibt dagegen widerspruchsfrei auf die höheren Automatisierungsgrade anwendbar. Bei der Produkthaftung ergibt sich bei einer automatisierten Fahrt, dass jeder Schaden, der nicht auf das Fehlverhalten eines Dritten zurückzuführen ist, potentiell zu einem Fall von Produkthaftung führt. Das könnte für die Hersteller enorme Folgen haben. Hier muss also auch noch nach Lösungen gesucht werden \parencite[S.6-7]{Gasser.2011}. Auch \Textcite[S.8]{Mages.2008} stellt fest, dass nicht vom Fahrer übersteuerbare Systeme grundlegende Änderungen des Straßenverkehrsrechts erfordern.