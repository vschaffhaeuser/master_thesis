% !TeX root = ../main.tex
% Add the above to each chapter to make compiling the PDF easier in some editors.

\chapter{Methodik}\label{chapter:Methodik}
%Kurz erklären, warum Hypothesen aufstestellt wurden. Und wofür die Skala oder Diagramm, je nachdem was es wird dienen soll.

\section{Hypothesen}\label{section:Hypothesen} %Unten und in Kapitel zwei bis jetzt nur Thesen. Was ist der Unterschied zwischen Thesen und Hypothesen? https://www.vwl.uni-mannheim.de/media/Fakultaeten Ich hab dann wohl Hypothesen aufgestellt?

Anhand der vorausgegangenen Literaturrecherche kann eine Vielzahl an Hypothesen bezüglich Situationen und Eigenschaften, die ein höheres Unfallrisiko mit sich bringen aufgestellt werden. Im Rahmen dieser Arbeit werden Hypothesen aufgestellt, die später mit den vorliegenden Unfalldaten verglichen und auf ihre Gültigkeit überprüft werden. Im Fokus vieler Unfallforschungen steht vielmals das Alter oder die Aufmerksamkeit der Unfallbeteiligten, ebenso spielt die Geschwindigkeitsüberschreitung häufig eine wichtige Rolle. Diese Punkte werden hier nicht weiter beachtet, da sie bei den vorhandenen Unfalldaten nicht gegeben sind. Des weiteren wird der Einfluss von berauschenden Mitteln nicht weiter analysiert. Dieser Punkt wird vernachlässigt, da später kritische Situationen mit einem menschlichen Fahrer, mit automatisierten Fahrsituationen verglichen werden sollen. Der Zustand des Fahrers ist beim vollautomatisierten Fahren nicht von Bedeutung und wird im Folgenden nicht weiter berücksichtigt. Um Unfälle kritischen Fahrsituationen zuordnen zu können muss man die Unfälle detailliert betrachten. Der Unfalltyp allein reicht dafür z.B. nicht aus. Deshalb sind auch die Hypothesen zum Teil sehr spezifisch und orientieren sich größtenteils an beim Unfall von der Polizei angegebenen Unfallursachen.

Das innerstädtische Straßennetz wird von Knotenpunkten geprägt, an denen viele verschiedene Fahrsituationen auftreten. Bei Abbiegevorgängen handelt es sich dabei um Situationen die viele Konfliktpunkte aufweisen. Es wir angenommen, dass:

\begin{itemize}
	\item Bei der Fahrsituation Linksabbiegen im Vergleich zur Situation Abbiegen nach rechts mehr Konfliktpunkte auftreten. Daher kommt es beim Linksabbiegen häufiger zu Unfällen. (\textit{Hypothese 1})
	\item Die Konfliktpunkte und Unfallzahlen können reduziert werden, wenn Linksabbieger, an Kreuzungen mit LSA, auf einem eigenen Fahrstreifen mit eigener Signalphase geführt werden. (\textit{Hypothese 2})
\end{itemize}

Es kommt im urbanen Raum jedoch nicht nur zu Unfällen durch Abbiegemanöver. Vor allem bei dichtem Verkehr dürfen Unfälle im Längsverkehr und mit Fahrzeugen im ruhenden Verkehr nicht vernachlässigt werden. 

\begin{itemize}	
	\item Bei höherem Verkehrsaufkommen, z.B. in den Hauptverkehrszeiten, steigt die Anzahl der Verkehrsunfälle im Längsverkehr. Ursachen dafür sind Konflikte beim Spurwechsel und zu geringer Sicherheitsabstand. (\textit{Hypothese 3})
	\item Im urbanen Raum kommt es häufig zu Konflikten mit Fahrzeugen im ruhenden Verkehr. Besonders auffällig sind Bereiche mit Längsaufstellung am Fahrbahnrand. Zusätzlich spielt verbotswidriges auf der Straße Halten/Parken, z.B. Parken in zweiter Reihe, eine bedeutende Rolle. Beim Vorbeifahren entstehen kritische Situationen die zu Unfällen führen. (\textit{Hypothese 4})
\end{itemize}

Urbane Fahrsituationen werden zusätzlich von nicht motorisierten Verkehrsteilnehmern beeinflusst. Hierbei muss berücksichtigt werden, dass:
	
\begin{itemize}
	\item Die Komplexität einer Fahrsituation und somit auch die Zahl der Unfälle wird erhöht, wenn nicht motorisierte Verkehrsteilnehmer daran beteiligt sind. Durch den geringen Schutz von Radfahrern/Fußgängern ist der Verletzungsgrad höher als bei Unfällen, an denen nur motorisierte Verkehrsteilnehmer beteiligt sind. (\textit{Hypothese 5})
	\item Wenn sich Fußgänger und Radfahrer parallel zum Fahrzeug bewegen kommt es beim Rechtsabbiegen häufiger zu Unfällen als beim Linksabbiegen. Kritische Situationen treten vor allem dann auf, wenn Radfahrer den Radweg in die falsch Richtung befahren. (\textit{Hypothese 6})
	\item Bei baulich von der Fahrbahn getrennten Radverkehrsanlagen kommt es häufiger zu Unfällen als bei Radverkehrsanlagen auf der Fahrbahn. Die Unfallgefahr wird durch schlechte bzw. nicht vorhandene Markierungen der Radverkehrsanlagen, besonders im Bereich von Knotenpunkten, erhöht. (\textit{Hypothese 7})
	\item Falsches Verhalten der Fußgänger ist häufig die Ursache für Unfälle mit Personenschaden im urbanen Raum. Besonders häufig treten die Ursachen Rotlichtverstöße und Überschreiten der Fahrbahn ohne auf den Fzg.-Verkehr zu achten auf. Häufig ereignen sich solche Unfälle in der Nähe von Haltestellen des ÖPNV's. (\textit{Hypothese 8})
\end{itemize}	

Neben den Unfallursachen, die Fahrzeugführern oder Fußgängern zugeschrieben werden können gibt es auch noch allgemeine Ursachen die von den Straßenverhältnissen, Witterungseinflüssen oder Hindernissen im Straßenraum beeinflusst werden. Hier kommt es vor allem 

\begin{itemize}	
	\item Bei Sichtbehinderungen durch Witterungseinflüsse bei der Unfallursache "Blendende Sonne" vermehrt zu Unfällen. (\textit{Hypothese 9})
\end{itemize}

Die hier aufgeführten Hypothesen werden in Kapitel \ref{sechtion:Überprüfung der Thesen} anhand der Unfalldaten des Testgebiets überprüft. Hier soll jedoch schon darauf hingewiesen werden, dass die Ergebnisse aufgrund der eher geringen Unfallzahlen im Testgebiet und teilweise unvollständigen Daten nicht auf andere Bereiche übertragbar sind. % Kann man das so stehen lassen?


\section{Kritikalitätsskala zur Bewertung von urbanen Fahrsituationen}
%Hier schon die Skala Entwickeln und in Kapitel vier anwenden. Kapitel fünf kann dann wegfallen. Wie kommt man anhand der Unfalldaten von bestimmten Unfalltypen zu den Fahrsituationen?? Schritt für Schritt erklären, dann muss ich nicht alle Unfälle mit den Kurzbeschreibungen auswerten. 

%Verkehrsunfälle stellen seltene Ereignisse dar und unterliegen wenn man sie statistisch betrachtet einer Poisson-Verteilung \parencite[S.18]{Grundl.2005}. Lambda wird dabei, als einziger Parameter der Poisson-Verteilung, als Risiko interpretiert. Das Risiko ist für jeden Verkehrsteilnehmer gleich und inhärent. Vorne schon in der Definition nur nochmal zu Info für mich!

%Ein möglicher Ansatz wäre die Risikoverteilung von Gschwendtner zu verwenden. Hier wird selten eine Höhe des Schadens anggegeben, wie können Unfälle mit Personenschaden eingestuft werden? Wert für leicht Verletzt, schwer Verletzt und Getötet? Es werden die detaillierten Unfalltypen verwendet, diese liegen bei den vorhanden Unfalldaten nicht vor. Evtl. nicht Unfalltyp sondern Unfallursache verwenden? Kommt dann näher an die detaillierten Unfalltypen hin. Oder schauen, ob es gelingt mit der Excel zu den Fahrsituationen anhand des Unfalltyp und der Unfallursache, die detailierten Fahrsituationen herauszufinden. Schaden evtl Kleinunfall, Sachschaden, Personenschaden (Personen nochmal unterteilen in leicht schwer und getötet)
%Um die Risikoäquivalenten besser verstehen zu können Paper von ihm lesen! Ist schon bei gelesen drin.





%Risikograph nach DIN V 19250 Auf S. 53 bei Hillenbrand wird in einer Tabelle dargestellt, welchen SIL die Buchstaben a bis h des Risikographs entsprechen. Brauch ich das hier? Wenn ich den Graph verwende muss ich ihn eh anpassen. Z.B. Schaden: Sachschaden, schwerwiegender Sachschaden, Personenschaden. Personen dann wo wie Kategorie A weiter in Leichtverletzt, Schwerverletzt, Getötet. Entrittswahrscheinlichkeit evtl. in 4 Abschnitte teilen kleine 25% kleiner 50% kleiner 75% und größer 75%. Der Betrachteten Fahrsituationen führten zu einem Unfall. !!!! Viele Unfälle nur mit Sachschaden würden trotzdem bei a landen. Muss noch überdacht werden.