% !TeX root = ../main.tex
% Add the above to each chapter to make compiling the PDF easier in some editors.

\subsection{Unfälle in München} %Zahlen beziehen sich auf die ganze Länge der drei Straßen.
Die Unfälle im gesamten Stadtgebiet München haben sich in den Jahren 2012 bis 2016 verändert. Die Anzahl der Unfälle mit Personenschaden ist bis zum Jahr 2014 kontinuierlich angestiegen, im Jahr 2015 nahmen die Zahlen wieder ab und erreichten 2016 sogar einen niedrigeren Wert als im Jahr 2012. Der Rückgang an Unfällen mit Personenschaden setzt sich im Jahr 2017 fort. Die Unfallzahlen der Leopoldstraße weisen ein ähnliches Bild auf, sie stiegen bis zum Jahr 2015 an und waren in den Jahren 2016 und 2017 rückläufig. Für die Schenkendorfstraße liegen nur wenige Unfallzahlen vor, was es erschwert einen Trend zu erkennen. Im Schnitt waren allerdings auch hier die Unfälle in den Jahren 2012 bis 2015 höher als in den zwei darauffolgenden Jahren. Die Unfallzahlen der Ungererstraße schwanken sehr stark. Sie hatten den höchsten Wert im Jahr 2014, gingen dann zurück sind im Jahr 2017 allerdings wieder angestiegen. Die genaue Anzahl der Unfälle mit Personenschaden im Stadtgebiet München und auf den drei Straßen können der Tabelle \ref{tab:Unfälle München Personenschaden} entnommen werden.

\begin{table}[htpb]
	\scriptsize
	\caption[Unfälle mit Personenschaden]{Unfälle mit Personenschaden der Jahre 2012 bis 2017 im gesamten Stadtgebiet der Stadt München und auf den drei Straßen im Testgebiet.}\label{tab:Unfälle München Personenschaden}
	\centering
	\begin{tabular}{l l l  l p{3cm}}
		\toprule
		Jahr & München & Leopoldstraße & Schenkendorfstraße & Ungererstraße \\
		\midrule
		2017 & 5290\footnotemark[1] & 77 & 12 & 26\\
		2016 & 5510\footnotemark[2] & 97 & 13 & 21\\
		2015 & 5634\footnotemark[3] & 103 & 17 & 33\\
		2014 & 5638\footnotemark[4] & 96 & 16 & 35\\
		2013 & 5584\footnotemark[5] & 88 & 12 & 22\\
		2012 & 5516\footnotemark[6] & 86 & 17 & 29\\
		\bottomrule
	\end{tabular}
\end{table}

Ein anderes Bild ergibt sich, wenn man die Unfälle mit schwerem Sachschaden betrachtet. Hier ist die Anzahl der Unfälle in München von 2012 bis 2014 rückläufig, steigt im Jahr 2015 wieder an und geht dann bis zum Jahr 2017 erneut zurück. Auf der Leopoldstraße schwankt die Anzahl der Unfälle, die meisten ereigneten sich im Jahr 2013, dann gingen sie bis zum Jahr 2015 zurück, stiegen im Jahr 2016 nochmal leicht an und erreichten 2017 dann wieder den Wert von 2015. Die Werte auf der Ungererstraße stagnierten zum Teil, waren aber im Verlauf der Jahre Rückläufig. Die Ungererstraße weist ebenfalls 2013 die größte Anzahl an Unfällen auf, diese gehen dann bis zum Jahr 2016 zurück, sind im Jahr 2017 jedoch wieder angestiegen. Einen Überblick über die genauen Unfallzahlen mit schwerwiegendem Sachschaden kann Tabelle \ref{tab:Unfälle München schwerw. Sachschaden} entnommen werden. 

\begin{table}[htpb]
	\scriptsize
	\caption[Unfälle mit schwerwiegendem Sachschaden]{Unfälle mit schwerwiegendem Sachschaden der Jahre 2012 bis 2017 im gesamten Stadtgebiet der Stadt München und auf den drei Straßen im Testgebiet.}\label{tab:Unfälle München schwerw. Sachschaden}
	\centering
	\begin{tabular}{l l l l p{3cm}}
		\toprule
		Jahr & München & Leopoldstraße & Schenkendorfstraße & Ungererstraße \\
		\midrule
		2017 & 518\footnotemark[1] & 121 & 15 & 45\\
		2016 & 690\footnotemark[2] & 124 & 17 & 32\\
		2015 & 761\footnotemark[3] & 121 & 17 & 42\\
		2014 & 713\footnotemark[4] & 125 & 17 & 45\\
		2013 & 833\footnotemark[5] & 135 & 19 & 64\\
		2012 & 839\footnotemark[6] & 125 & 19 & 63\\
		\bottomrule
	\end{tabular}
\end{table}

\section{Thesen}
\begin{itemize}
	%\item Sichtbehinderungen und schlechte Einsehbarkeit an Knotenpunkten können eine häufige Ursache für Unfälle sein, da sie bestimmte Fahrsituationen erschweren. 
	%\item An Knotenpunkten mit Lichtsignalanlage (LSA) ist die Anzahl an Unfällen höher als an Knotenpunkten ohne LSA. Wird die LSA in der Nacht ausgeschaltet steigt das Unfallrisiko.
	%\item An Kreuzungen treten im Vergleich zu Einmündungen häufiger Unfälle auf.
	%\item Bei der Fahrsituation Linksabbiegen kommt es zu den meisten potentiellen Konfliktpunkten, hier treten daher häufig Unfälle auf. Ein separater Linksabbiegestreifen kann die Konfliktpunkte reduzieren. 
	%\item Bei höherem Verkehrsaufkommen steigt die Anzahl der Unfälle. Besonders betroffen sind die Unfälle im Längsverkehr.
	%\item Regelverstöße lösen Konfliktsituationen aus und führen zu Unfällen.
	%\item Sind nicht motorisierte Verkehrsteilnehmer an einer Fahrsituation beteiligt wird die Komplexität erhöht und es kommt vermehrt zu Unfällen.
	%\item Bei Fußgängern und Radfahrern ist die Anzahl an Rotlichtverstößen höher als bei motorisierten Verkehrsteilnehmern.
	%\item Alkoholisierte Fußgänger und Radfahrer sind häufiger am Unfallgeschehen beteiligt als alkoholisierte Autofahrer.
	%\item Schlechte bzw. nicht vorhandene Markierungen an Fußgänger- und Fahrradüberwegen erhöhen die Unfallgefahr.
	%\item Wenn sich Fußgänger und Radfahrer parallel zum Fahrzeug bewegen steigt die Unfallgefahr beim Abbiegen. Noch höher ist das Risiko, wenn Radfahrer dabei den Radweg in die falsche Richtung befahren.
	%\item Bei baulich von der Fahrbahn getrennten Radverkehrsanlagen kommt es häufiger zu Unfällen als bei Radverkehrsanlagen auf der Fahrbahn. 
	%\item Der Straßenzustand hat einen erheblichen Einfluss auf das Unfallgeschehen.
\end{itemize}
