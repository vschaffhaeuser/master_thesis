% !TeX root = ../main.tex
% Add the above to each chapter to make compiling the PDF easier in some editors.

\chapter{Erstellung einer Bewertung von Fahrsituationen bzgl. ihrer Sicherheitsrelevanz}\label{chapter:Kritikalitätsskala}

%Wird in Kapitel 3 und 4 schon berücksichtigt. Kapiel 5 fällt deshalb weg.

%Ein möglicher Ansatz wäre die Risikoverteilung von Gschwendtner zu verwenden. Hier wird selten eine Höhe des Schadens anggegeben, wie können Unfälle mit Personenschaden eingestuft werden? Wert für leicht Verletzt, schwer Verletzt und Getötet? Es werden die detaillierten Unfalltypen verwendet, diese liegen bei den vorhanden Unfalldaten nicht vor. Evtl. nicht Unfalltyp sondern Unfallursache verwenden? Kommt dann näher an die detaillierten Unfalltypen hin. Oder schauen, ob es gelingt mit der Excel zu den Fahrsituationen anhand des Unfalltyp und der Unfallursache, die detailierten Fahrsituationen herauszufinden. Schaden evtl Kleinunfall, Sachschaden, Personenschaden (Personen nochmal unterteilen in leicht schwer und getötet)
%Um die Risikoäquivalenten besser verstehen zu können Paper von ihm lesen! Ist schon bei gelesen drin.

%Risikograph nach DIN V 19250 Auf S. 53 bei Hillenbrand wird in einer Tabelle dargestellt, welchen SIL die Buchstaben a bis h des Risikographs entsprechen. Brauch ich das hier? Wenn ich den Graph verwende muss ich ihn eh anpassen. Z.B. Schaden: Sachschaden, schwerwiegender Sachschaden, Personenschaden. Personen dann wo wie Kategorie A weiter in Leichtverletzt, Schwerverletzt, Getötet. Entrittswahrscheinlichkeit evtl. in 4 Abschnitte teilen kleine 25% kleiner 50% kleiner 75% und größer 75%. Der Betrachteten Fahrsituationen führten zu einem Unfall. !!!! Viele Unfälle nur mit Sachschaden würden trotzdem bei a landen. Muss noch überdacht werden.