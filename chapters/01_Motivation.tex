% !TeX root = ../main.tex
% Add the above to each chapter to make compiling the PDF easier in some editors.

\chapter{Einleitung}\label{chapter:Motivation}
% vgl. Dis. Gerstenberger
% Beispiel für ein Zitat: \parencite[][417--425]{AbdelAty.2005}

Die Verkehrssicherheit auf deutschen Straßen ist in den letzten Jahren immer wieder in den Fokus der Öffentlichkeit geraten und spielt eine wichtige Rolle. Die Sicherheit betreffend werden sowohl Infrastrukturmaßnahmen als auch Maßnahmen an den Fahrzeugen selbst diskutiert. Da über 90\% \parencite[S.48]{DEKRA.2017} der Verkehrsunfälle durch menschliche Fehler verursacht werden liegt der Fokus aktuell auf der Einführung und Entwicklung unterstützender Fahrerassistenzsystemen (FAS). Hier ist es bereits gelungen, dass dieses Systeme zum Teil schon serienreif sind und zunehmend Einzug in die Fahrzeuge nehmen. Bereits vorhandenen Systeme adressieren überwiegend Fahrsituationen in Außerortsbereichen, der urbane Raum wurde bisher nur in wenigen Fällen betrachtet und bietet daher erhöhtes Potential die Verkehrssicherheit zu verbessern. Zudem sind urbane Fahrsituationen weitaus komplexer als Situationen auf Autobahnen und Landstraßen weshalb sie häufiger zu Unfällen führen. Eine weitere Vision ist es Unfälle, durch automatisiertes Fahren, komplett zu verhindern. 

\section{Motivation}

Die Zahl der Verkehrstoten auf deutschen Straßen war im Jahr 2016 so gering wie noch nie seit Beginn der Aufzeichnungen \parencite[S.5]{StatistischesBundesamt.2018}. Bei der gesamt Anzahl an Verkehrsunfällen ist jedoch kein Rückgang zu verzeichnen. Es kommt zwar seltener zu Unfällen mit Getöteten, dafür steigt die Anzahl der Unfälle mit Personenschaden und Sachschaden \parencite[S.5]{StatistischesBundesamt.2018}. Dies kann zum einen an der steigenden Zahl zugelassener Pkw liegen \parencite[S.5]{StatistischesBundesamt.2018}, zum anderen an der zunehmenden Urbanisierung. Die Urbanisierung führt dazu, dass immer mehr Menschen in die Städte ziehen, was zu einer steigenden Verkehrsdichte führt. In den Städten kommt es daher häufiger zu Konfliktsituationen, meist mit geringeren Geschwindigkeiten, die zu Unfällen führen.

Besonders an Knotenpunkten kommt es innerorts häufig zu Unfällen, 2017 ereigneten sich fast 16\% der Unfälle mit Personen- und schwerwiegendem Sachschaden beim Abbiegen und 28\% beim Einbiegen/Kreuzen \parencite[S.68]{StatistischesBundesamt.2018b}. Fahrsituationen an Knotenpunkten sind komplex und führen häufig zu kritischen Situationen, da der Fahrer nicht alle Informationen aufnehmen kann \parencite[S.2]{Gerstenberger.17.02.2015} und das durchfahren eines Knotenpunktes durch ein komplexes Zusammenspiel von vielen Einzelaufgaben gekennzeichnet ist \parencite[S.51]{Zademach.24.09.2015}. Diese Situationen bieten ein besonders hohes Potential bei der Entwicklung von FAS oder automatisierten Fahrzeugen. Gelingt es anhand von Einrichtungen im Fahrzeug komplexe Situationen schnell und richtig zu erkennen kann dies den Fahrer unterstützen bzw. ihn komplett von seiner Fahraufgabe befreien und die Unfallzahlen können reduziert werden. 

Aufgrund der Komplexität von innerstädtischen Verkehrssituationen gibt es aktuell noch keine serienreifen Systeme, es sind jedoch viele verschiedene Ansätze und Prototypen zu erkennen. Bei der Entwicklung spielt die Unfallforschung eine wichtige Rolle. Anhand existierender Unfalldaten kann herausgearbeitet werden, in welchen Situationen es häufig zu Unfällen kommt und was die Ursachen für diese Unfälle waren. Häufig stehen jedoch nur die Endpositionen der Fahrzeuge und Aussagen der Fahrer zur Verfügung und der Verlauf des Unfalls muss mühsam rekonstruiert werden. Ebenso gibt es eine hohe Anzahl an Unfällen, die nicht bei der Polizei gemeldet werden und somit nicht in die Unfallstatistiken mit einfließen. Der Faktor des Dunkelfelds der nicht polizeilich erfassten Unfälle befindet sich zwischen drei und zehn \parencite[S.151]{Huguenin.2017}.

\section{Zielstellung}
%Nochmal durchgehen! %Literaturrecherche erwähnen?Nachfolgendes Zitat einbinden.

\enquote{Um einen Unfall verhindern zu können, ist es wesentlich zu verstehen, wie die entsprechende Situation unfallfrei bewältigt werden kann} \parencite[S.8]{Vollrath.2006}.

Das Ziel dieser Arbeit stellt die Bewertung unterschiedlicher Fahrsituationen auf einer Testroute im Münchner Norden im Hinblick auf ihre Sicherheitsrelevanz dar. Hierzu werden Unfalldaten, die von der Polizei für die Leopoldstraße, Ungererstraße und Schenkendorfstraße zur Verfügung gestellt wurden, aufbereitet und ausgewertet. Die Daten reichen über einen Zeitraum von fünf Jahren (2012 bis 2016) und dienen als Grundlage für die Ermittlung von Stellen und Fahrsituationen die häufig zu Unfällen führen. Die markanten Punkte im Netz werden genauer betrachtet, um ein besseres Verständnis über die auftretenden Konfliktsituationen, die zu Unfällen führen, zu erhalten. Die Ergebnisse der Unfalldatenanalyse werden den Fahrsituationen im Testgebiet gegenübergestellt. Die Gegenüberstellung ermöglicht eine Bewertung der Situationen bezüglich ihrer Sicherheitsrelevanz.

Fahrsituationen die besonders sicherheitskritisch sind sollen bei der Entwicklung von automatisierten Fahrzeugen priorisiert und noch genauer analysiert werden. Im Hinblick auf die Entwicklung automatisierter Fahrzeuge werden die ermittelten sicherheitsrelevanten Fahrsituationen für menschliche Fahrer mit denen für automatisierte Fahrzeuge abgeglichen. Die daraus gewonnenen Erkenntnisse sollen bei der Entwicklung von automatisierten Fahrzeugen helfen und zur Steigerung der Verkehrssicherheit sowie zur Reduzierung von Verkehrsunfällen beitragen. 

\section{Struktur der Arbeit}
% Ganz am Ende einen kurzen Überblick daarüber geben, wie die Arbeit aufgebaut ist.