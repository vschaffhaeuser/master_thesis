% !TeX root = ../main.tex
% Add the above to each chapter to make compiling the PDF easier in some editors.

\chapter{Einleitung}\label{chapter:Motivation}
Die Verkehrssicherheit auf deutschen Straßen ist in den letzten Jahren immer wieder in den Fokus der Öffentlichkeit geraten und spielt eine wichtige Rolle. Die Sicherheit betreffend werden sowohl Infrastrukturmaßnahmen als auch Maßnahmen an den Fahrzeugen selbst diskutiert. Da über 90 \% \parencite[S. 48]{DEKRA.2017} der Verkehrsunfälle durch menschliche Fehler verursacht werden liegt der Fokus aktuell auf der Einführung und Entwicklung von unterstützenden Fahrerassistenzsystemen (FAS). Hier ist es bereits gelungen, einige serienreife  Systeme zu entwickeln die zunehmend Einzug in die Fahrzeuge nehmen. Bereits vorhandenen Systeme adressieren überwiegend Fahrsituationen in Außerortsbereichen. Der urbane Raum wurde bisher nur in wenigen Fällen betrachtet und bietet daher ein erhöhtes Potential, die Verkehrssicherheit zu verbessern. Zudem führen urbane Fahrsituationen aufgrund weitaus komplexeren Situationen, im Vergleich zu Situationen auf Autobahnen und Landstraßen, häufiger zu Unfällen. Die Vision Unfälle komplett zu verhindern zu können existiert schon lange. Die Entwicklung automatisierte Systeme soll hierbei einen wesentlichen Beitrag leisten. 

\section{Motivation}
Die Zahl der Verkehrstoten auf deutschen Straßen war im Jahr 2016 so gering wie noch nie seit Beginn der Aufzeichnungen \parencite[S. 5]{StatistischesBundesamt.2018}. Bei der Gesamtanzahl an Verkehrsunfällen ist jedoch kein Rückgang zu verzeichnen. Es kommt zwar seltener zu Unfällen mit Getöteten, dafür steigt die Anzahl der Unfälle mit Personenschaden und Sachschaden \parencite[S. 5]{StatistischesBundesamt.2018}. Dies kann zum einen an der steigenden Zahl zugelassener Pkw liegen \parencite[S. 5]{StatistischesBundesamt.2018}, zum anderen an der zunehmenden Urbanisierung. Die Urbanisierung führt dazu, dass immer mehr Menschen in die Städte ziehen, was eine steigenden Verkehrsdichte zur Folge hat. In den Städten kommt es daher häufiger zu Konfliktsituationen, meist mit geringeren Geschwindigkeiten, die zu Unfällen führen.

Besonders an Knotenpunkten kommt es innerorts häufig zu Unfällen, 2017 ereigneten sich fast 16 \% der Unfälle mit Personen- und schwerwiegendem Sachschaden beim Abbiegen und 28 \% beim Einbiegen/Kreuzen \parencite[S. 68]{StatistischesBundesamt.2018b}. Fahrsituationen an Knotenpunkten sind komplex und führen häufig zu kritischen Situationen, da der Fahrer nicht alle Informationen aufnehmen kann \parencite[S. 2]{Gerstenberger.17.02.2015} und das durchfahren eines Knotenpunktes durch ein komplexes Zusammenspiel von vielen Einzelaufgaben gekennzeichnet ist \parencite[S. 51]{Zademach.24.09.2015}. Diese Situationen bieten ein besonders hohes Potential bei der Entwicklung von FAS oder automatisierten Fahrzeugen. Gelingt es anhand von Einrichtungen im Fahrzeug komplexe Situationen schnell und richtig zu erkennen kann dies den Fahrer unterstützen bzw. ihn komplett von seiner Fahraufgabe befreien und die Unfallzahlen können reduziert werden. 

Aufgrund der Komplexität von innerstädtischen Verkehrssituationen gibt es aktuell noch keine serienreifen Systeme die für Fahrsituationen im urbanen Raum geeignet sind. Es sind jedoch viele verschiedene Ansätze und Prototypen in der  Literatur zu erkennen. Bei der Entwicklung spielt die Unfallforschung eine wichtige Rolle. Anhand existierender Unfalldaten kann herausgearbeitet werden in welchen Situationen es häufig zu Unfällen kommt und was die Ursachen für diese Unfälle waren. Häufig stehen jedoch nur die Endpositionen der Fahrzeuge und Aussagen der Fahrer zur Verfügung. Der Verlauf des Unfalls hingegen muss mühsam rekonstruiert werden. Ebenso gibt es eine hohe Anzahl an Unfällen die nicht bei der Polizei gemeldet werden und somit nicht in die Unfallstatistiken mit einfließen. Der Faktor des Dunkelfelds der nicht polizeilich erfassten Unfälle befindet sich zwischen drei und zehn \parencite[S. 151]{Huguenin.2017}.

\section{Zielstellung}
\enquote{Um einen Unfall verhindern zu können, ist es wesentlich zu verstehen, wie die entsprechende Situation unfallfrei bewältigt werden kann} \parencite[S. 8]{Vollrath.2006}.

Das oben genannte Zitat geht auf einen wesentlichen Punkt dieser Arbeit ein. Ziel dieser Arbeit ist die Bewertung unterschiedlicher Fahrsituationen auf einer Testroute im Münchner Norden. Die Situationen werden bezüglich ihrer Sicherheitsrelevanz bewertet, so ist es möglich kritische Situationen zu verstehen und mögliche Unfallursachen ausfindig zu machen. Hierzu werden Unfalldaten, die vom Polizeipräsidium München für die Leopoldstraße, Ungererstraße und Schenkendorfstraße zur Verfügung gestellt wurden, aufbereitet und ausgewertet. Die Daten reichen über einen Zeitraum von fünf Jahren (2012 bis 2016) und dienen als Grundlage für die Ermittlung von Örtlichkeiten und Fahrsituationen die häufig zu Unfällen führen. Die markanten Punkte im Netz werden genauer betrachtet, um ein besseres Verständnis über die auftretenden Konfliktsituationen, die zu Unfällen führen, zu erhalten. Die Ergebnisse der Unfalldatenanalyse werden dann den Fahrsituationen im Testgebiet gegenübergestellt. Anhand der Gegenüberstellung wird eine Bewertung der Situationen bezüglich ihrer Sicherheitsrelevanz ermöglicht.

Fahrsituationen die ein erhöhtes Risiko aufweisen sollen bei der Entwicklung von automatisierten Fahrzeugen priorisiert und noch genauer analysiert werden. Im Hinblick auf die Entwicklung automatisierter Fahrzeuge werden die ermittelten, sicherheitsrelevanten, Fahrsituationen für menschliche Fahrer mit denen für automatisierte Fahrzeuge verglichen. Die daraus gewonnenen Erkenntnisse sollen bei der Entwicklung von automatisierten Fahrzeugen helfen und zur Steigerung der Verkehrssicherheit, sowie zur Reduzierung von Verkehrsunfällen beitragen. 

\section{Struktur der Arbeit}
Nach den einleitenden Worten und Angaben zu Ziel und Aufbau der Arbeit in Kapitel \ref{chapter:Motivation} werden in Kapitel \ref{chapter:Datenauswertung} zunächst wichtige Begrifflichkeiten für das Verständnis der Arbeit anhand von Definitionen eingeführt. Zudem gibt dieses Kapitel einen allgemeinen Überblick über Verkehrsunfälle in Deutschland. Anschließend menschliche Reaktionen und Handlungen im Straßenverkehr berücksichtigt. Da urbane Fahrsituationen ein hohes Unfallpotenzial bieten werden diese noch detaillierter betrachtet, bevor abschließend mögliche Ansätze zur Bewertung von Fahrsituationen vorgestellt.

Mit der Methodik die im weiteren Verlauf der Arbeit angewendet wird befasst sich Kapitel \ref{chapter:Methodik}. Zu Beginn werden Hypothesen aufgestellt, die im Folgenden Kapitel überprüft werden. Ein weiterer Punkt bildet die Entwicklung einer Bewertungsmethodik für Unfälle und Fahrsituationen im urbanen Raum.

Das Unfallgeschehen im Untersuchungsgebiet wird in Kapitel \ref{chapter:Datenauswertung} anhand eines umfangreichen Unfalldatensatzes analysiert. Die Hypothesen aus Kapitel \ref{chapter:Methodik} werden überprüft und die Häufigkeit und Unfallschwere von bestimmten Unfalltypen ermittelt. Um kritische Fahrsituationen zu ermitteln werden die Unfälle kategorisiert und dann möglichen Fahrsituationen innerhalb des Testgebiets zugeordnet.

Kritische Situationen, die sich aus dem vorangehenden Kapitel ergeben, werden in Kapitel \ref{chapter:automatisiertes Fahren} mit automatisierten Systemen verglichen. Es wird aufgezeigt in welchen Fällen diese Potenzial besitzen Unfälle zu vermeiden bzw. in was für Situationen auch automatisierte Systeme Schwierigkeiten haben könnten. 

Abschließend werden in Kapitel \ref{chapter:Diskussion} die wichtigsten Ergebnisse dieser Arbeit nochmals zusammengefasst und Ansatzpunkte aufgezeigt, die weiteren Forschungsbedarf benötigen.