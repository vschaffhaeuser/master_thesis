% !TeX root = ../main.tex
% Add the above to each chapter to make compiling the PDF easier in some editors.

\chapter{Fazit und Ausblick}\label{chapter:Diskussion}
%Fazit
Die vorausgehende Datenauswertung verdeutlicht, dass Unfälle und Fahrsituationen im urbanen Raum sehr vielfältig sind. Nicht nur Unfälle, bei denen es zu Schwerverletzten oder Getöteten kommt, bergen ein hohes Risiko in sich. Auch Fahrsituationen, die auf den ersten Blick eher harmlos erscheinen, kann ein hohes Risiko zugeordnet werden. Innerhalb des Testgebiets kommt dieser Fall bei der Situation mit dem Feintyp 501 vor. Konflikte mit Fahrzeugen im fließenden Verkehr und parkenden Fahrzeugen auf Längsparkplätzen ereigneten sich so häufig, dass sie mit der Risiko-Kategorie \textit{d} bewertet wurden. Folgt aus solch einer Fahrsituation ein Unfall, ist die Ursache meist ein einfacher menschlicher Fehler, z.B. Unaufmerksamkeit. Fahrsituationen, die anfällig für menschliche Fehler sind, bieten ein hohes Potential für automatisierte Systeme. Automatisierung kann voraussichtlich vor allem die aktuell hohe Anzahl der Kleinunfälle reduzieren.

Fahrsituationen, an denen nicht motorisierte Verkehrsteilnehmer beteiligt sind, sind komplexer und bringen meist ein erhöhtes Risiko mit sich. Dies liegt vor allem daran, dass Unfälle mit ungeschützten Verkehrsteilnehmern oft zu schwereren Verletzungen führen. Besonders häufig kommt es an Knotenpunkten zu Konflikten mit Fußgängern und Radfahrern. Des weiteren kommt es in Bereichen, die einen Anreiz zum Queren der Straße bieten (z.B. Geschäftsstraße, Haltestelle des ÖPNVs), zu Konflikten mit Fußgängern. Diese Konflikte können zum Teil vermieden werden, wenn automatisierte Systeme in der Lage sind, alle Verkehrsteilnehmer, die an der jeweiligen Fahrsituation beteiligt sind, zuverlässig zu erkennen. Sichtbehinderungen können der Grund dafür sein, dass eine zuverlässige Erkennung nicht immer möglich ist. Zudem müssen Informationen über die Handlungsabsichten der jeweiligen Verkehrsteilnehmer vorliegen. Automatisierte Systeme müssen in der Lage sein, mit nicht motorisierten Verkehrsteilnehmern zu kommunizieren, um Konfliktsituationen zu vermeiden.

Um Fahrsituationen klassifizieren zu können sind neben der Art der Verkehrsteilnahme auch die Charakteristik der Strecke sowie die Umfeldbedingungen relevant. Bereiche mit einem hohen Verkehrsaufkommen sind z.B. anfälliger für Unfälle im Längsverkehr. Innerhalb des Testgebiets ist dies besonders im Bereich der Schenkendorfstraße zu erkennen. Knotenpunkte sind allgemein anfällig für Unfälle, da häufig komplexe Fahrsituationen auftreten. Trotzdem ist es möglich, die Konflikte an Knotenpunkten durch deren Gestaltung zu reduzieren (z.B. eigene Signalphase für Linksabbieger). Umfeldbedingungen haben ebenfalls einen Einfluss auf das Unfallgeschehen. In Bereichen, in denen viele Einkaufsmöglichkeiten angesiedelt sind, kommt es z.B. häufiger zu Unfällen im ruhenden Verkehr. Bereiche mit stark frequentierten Haltestellen des ÖPNVs sind dagegen anfällig für Unfälle mit Fußgängern. Bei der Entwicklung von automatisierten Systemen sollte daher nicht nur die Fahrsituation selbst, sondern immer auch das entsprechende Gebiet, in dem die Situation auftritt, mit berücksichtigt werden, da viele Faktoren auf die Entstehung einer bestimmten Situation einwirken.

Automatisierte Systeme bieten die Möglichkeit, Fahrsituationen, die aufgrund von menschlichem Versagen zu einem Unfall führten, größtenteils zu vermeiden. Es treten jedoch auch hier Grenzen auf. Ein Beispiel hierfür ist, wenn Fußgänger unmittelbar vor dem Fahrzeug auf die Straße treten. Ab einem gewissen Punkt ist es physikalisch nicht mehr möglich, den Unfall zu verhindern, da selbst bei sofortiger Erkennung des Konflikts, die für ein Bremsmanöver zur Verfügung stehende Zeit zu gering ist. Zudem könnte es in der Einführungsphase, in der menschliche Fahrer und automatisierte Systeme gleichzeitig am Straßenverkehr teilnehmen, zu neuen Konfliktsituationen kommen, die sich aus Missverständnissen zwischen den Fahrern ergeben. Wichtig ist auch, dass automatisierte Systeme hohen Anforderungen entsprechen und dauerhaft zuverlässig funktionieren müssen. Die periodische Überwachung des Fahrzeugs gewinnt daher noch mehr Bedeutung \parencite[S. 62]{DEKRA.2017}.

%Ausblick
Um den hohen Anforderungen, die mit automatisierten Systemen einhergehen, gerecht zu werden, muss die Unfallanalyse in der Zukunft präzisiert werden. Probleme der Unfallanalysen sind, dass Unfälle seltene Ereignisse darstellen und eine Veränderung des Unfallgeschehens erst nach Ablauf einer relativ langen Erhebungsperiode ersichtlich ist. Zudem können Unfälle nicht vollständig und zuverlässig erfasst werden, die Dunkelziffer kann je nach Unfallart bis zu 85 \% betragen \parencite[S. 
7-9]{Erke.1978}. Oft werden Unfälle auch nicht ausreichend genau erfasst, wobei hier Projekte wie GIDAS den richtigen Ansatz liefern. Sie erfassen jedoch nur Unfälle mit Personenschaden genauer. Um mehr Informationen über das Unfallgeschehen zu erhalten, sind weitere solcher Projekte nötig, die sich auch mit Unfällen beschäftigen, die nur zu einem Sachschaden führen und sich überwiegend im niedrigen Geschwindigkeitsbereich ereignen.

Neben der detaillierten Unfallaufnahme in Projekten kann auch die Unfallaufnahme der Polizei erweitert werden. Bei der Datenanalyse in dieser Arbeit wäre es z.B. hilfreich gewesen, wenn für Kleinunfälle die gleichen Informationen wie für Unfälle mit Personen- und Sachschaden im engeren Sinne vorliegen würden. Hierfür müsste die bei diesen Unfällen durchgeführte Kurzaufnahme durch die normale Unfallaufnahme ersetzt werden. Es liegen zudem wenig Informationen vor, welcher menschliche Fehler zu einem Unfall geführt hat. Es wird zwar oft angegeben, dass der Fahrer abgelenkt war, der Grund für die Ablenkung wird jedoch nicht erwähnt. Dies ist schwer herauszufinden, da solche Angaben meist auf der Ehrlichkeit des Unfallfahrers selbst beruhen. Trotzdem wäre es interessant zu wissen, ob der Fahrer z.B. durch Handynutzung, Kommunikation mit einem Beifahrer oder durch zu viele evtl. schlecht lesbare Schilder abgelenkt war. Dies wäre auch im Bezug auf fahrerloses Fahren interessant, da auch automatisierte Systeme beim letzten Punkt Schwierigkeiten haben könnten. Sobald man die Unfallaufnahme erweitert, muss darauf geachtet werden, dass dies einheitlich und am besten bundesweit geschieht. Wichtig ist auch, dass Informationen, die sich nachträglich aus der Unfallrekonstruktion ergeben, in das Protokoll der Unfallaufnahme übertragen werden. Für diese Arbeit stand keine Unfallrekonstruktion der Polizei zur Verfügung. Sobald diese vorliegt können genauere Rückschlüsse auf den Unfallhergang gezogen werden. Es kann z.B. die Ausgangsgeschwindigkeit ermittelt werden, die Positionen der Unfallfahrzeuge sind bekannt und können mit dem Umfeld abgeglichen werden.

Um einfacher an Unfalldaten zu kommen, könnte man, ähnlich wie bei Flugzeugen, auch in Kraftfahrzeugen \ac{UDS} verbauen. Diese können nach einem Unfall Auskunft über die genaue Unfallzeit, die Geschwindigkeit, das Betätigen der Blinker und der Bremse sowie die Beschleunigung in Längs- und in Querrichtung zum Fahrzeug geben. Obwohl sie auch nachträglich eingebaut werden können, werden \ac{UDS} bis jetzt selten in Kfzs verwendet. Ein Grund dafür könnte die Angst der Fahrer sein, dass ihre Daten \enquote{ausspioniert} werden, was allerdings nicht der Fall ist. Unfalldatenschreiber messen zwar die ganze Zeit über Daten, aufgezeichnet werden die Daten allerdings nur unter bestimmten Bedingungen, die auf einen Unfall schließen lassen und dann auch nur über einen Zeitraum von ca. 45 s \parencite[S. 99]{Burg.2017}.

Während die \Textcite[S. 12]{DEKRA.2017} der Meinung ist, dass das reale Unfallgeschehen auf den Straßen die entscheidende Rolle für die Beurteilung der Verkehrssicherheit und die Einleitung entsprechender Optimierungsmaßnahmen spielt, legt \Textcite[S. 10]{Erke.1978} Wert auf die Verkehrskonflikterkennung. In der Verkehrskonflikterkennung werden nicht die Unfälle analysiert, sondern alle Konfliktsituationen, unabhängig davon, ob sich daraus ein Unfall ereignet oder nicht. Dies sollte auch die Grundlage für die Entwicklung automatisierter Fahrzeuge sein, da die Daten aus Unfallaufnahmen meist zu wenig Informationen preisgeben. Es muss bereits die Anzahl der Konflikte bzw. Konfliktsituationen reduziert werden. Ist nicht bekannt, welche Konfliktsituationen zu Unfällen führen, können diese bei der Entwicklung auch nicht berücksichtigt werden.

Die Statistischen Ämter des Bundes und der Länder stellen auf ihrer Homepage einen Unfallatlas zur Verfügung, auf den jeder frei zugreifen kann \parencite{StatistischeAmterdesBundesundderLander.2018}. Interessierte können nachsehen, auf welchen Strecken bzw. an welchen Punkten im Netz sich häufig Unfälle ereigneten. Diese Karte kann dazu beitragen, dass Verkehrsteilnehmer aufmerksamer am Verkehrsgeschehen teilnehmen, da ihnen das Risiko bestimmter Fahrsituationen bewusst ist, oder dass sie besonders riskante Bereiche meiden. Zudem kann man anhand der Karte auch ablesen, welche Situationen/Bereiche bei der Entwicklung automatisierter Systeme berücksichtigt werden müssen. Das Projekt UR:BAN ist ein weiteres Projekt, das hilfreich für die Entwicklung automatisierter Systeme sein könnte. Hier werden urbane Fahrsituationen in deutschen Städten betrachtet und analysiert. Ziel ist die Entwicklung von \ac{FAS} und Verkehrsmanagementsystemen \parencite{UR:BANBuro.2018}.

Zusammenfassend kann gesagt werden, dass die Unfallaufnahme und die Unfallforschung zukünftig weiter verbessert werden müssen, um Fahrsituationen, die zu Unfällen führen, besser identifizieren zu können. Zusätzlich ist es erforderlich, weitere Ansätze zu verfolgen, die sich nicht nur mit Unfällen befassen sondern Konfliktsituationen betrachten, auch wenn sich daraus kein Unfall ergibt. Des weiteren müssen die rechtlichen Bedingungen angepasst werden, um die Entwicklung von automatisierten Fahrzeugen voran zu treiben. Alles in allem birgt das automatisierte Fahren ein hohes Potential, Unfälle zu vermeiden. Die Vision, Unfälle komplett vermeiden zu können, wird jedoch voraussichtlich auch mit automatisierten Systemen in naher Zukunft nicht zu erfüllen sein.
