% !TeX root = ../main.tex
% Add the above to each chapter to make compiling the PDF easier in some editors.

\chapter{Fazit und Ausblick}\label{chapter:Diskussion}

"Es besteht ein dringender Handlungsbedarf in der Umsetzung wie in der Forschung zur Verkehrssicherheit." Ziel kann es auch sein, "mit ambitionierten Verkehrssicherheitszielen den Weg zu weisen: Towards Zero"
"Als wirksam erkannte Verkehrssicherheitsmaßnahmen sollten mit hoher Dringlichkeit umgesetzt werden."
"Die rechtlichen Bedingungen für die Einführung autonom wirkender Fahrerassistenzsysteme müssen verbessert werden." \parencite[S.170]{WissenschaflicherBeiratbeimBundesministerfurVerkehrBauundStadtentwicklung.2011b}

%S.146-147 von Gerstenberger, gibt die Situationen mit Unterstützungsbedarf und die Ursachen von Problemsituationen an, mit meinen Ergebnissen abgleichen.

%Erke verwendet die Verkehrskonflikt Technik. Hier werden nicht die Unfälle analysiert, sondern alle Konfliktsituationen, egal ob sich daraus ein Unfall ereignet oder nicht. \parencite[S.10]{Erke.1978} Für die Entwicklung automatisierter Fahrzeuge, sollte auch so vorgegangen werden. Die Daten aus Unfallaufnahmen geben zu wenig Infomationen preis. Die Anzahl der Konflikte muss reduziert werden. Wenn man nicht weiß aus was für Konflikten sich Unfälle ereignen könne kann, werden sie bei der Entwicklung nicht berücksichtigt. (Es werden auf S.10 noch mehr Vorteile genannt.)

%Bei Gründel stehen z.B. zur besseren Auswertung der Daten och die Unfallrekonstruktionen der Polizei zur Verfügung. Man kann so genauere Rückschlüsse auf den Unfallhergang ziehen. Es kann z.B. die Ausgangsgeschwindigkeit ermittelt werden, man weiß die Positionen des Fahrzeuge, diese können mit dem Umfeld abgeglichen werden. Was hat der Fahrer übersehen? Wurde ein Schild verdeckt? Erschweren Sichtbehinderungen die Sicht auf den Querenden Verkehr...

"Tatsache ist: Für die Beurteilungen der Verkehrssicherheit und die Einleitung entsprechender Optimierungsmaßnahmen spielt das reale Unfallgeschehen auf den Straßen die ganz entscheidende Rolle." \parencite[S.12]{DEKRA.2017}
"Auch in Bezug auf die Ursachenanalyse von Unfällen und Verletzungen bestehe nach wie vor ein großes Defizit." \parencite[S.14]{DEKRA.2017}

Eine Möglichkeit einfacher an Unfalldaten zu kommen ist es ähnlich wie in Flugzeugen auch in Kfz-Fahrzeugen Unfalldatenspeicher (UDS) zu verbauen. Diese können nach einem Unfall Auskunft über die genaue Unfallzeit, die Geschwindigkeit, das Betätigen der Blinker und der Bremse, sowie die Beschleunigung in Längs. und in Querrichtung zum Fahrzeug geben. Bis jetzt werden UDS, obwohl sich auch nachträglich eingebaut werden können, selten in Kfz-Fahrzeugen verwendet. Ein Grund hierfür könnte die Angst der Fahrer sein ihre Daten werden \enquote{ausspioniert}. Dies ist allerdings nicht der Fall, da Unfalldatenschreiber zwar die ganze Zeit über Daten messen, aufgezeichnet werden die Daten allerdings nur unter bestimmten Bedingungen, die auf einen Unfall schließen lassen und dann auch nur über einen Zeitraum von ca. 45 s \parencite[S.99]{Burg.2017}.

Probleme der Unfallanalyse sind, dass Unfälle seltene Ereignisse darstellen und eine Veränderung des Unfallgeschehens erst nach Ablauf einer relativ langen Erhebungsperiode ersichtlich sind. Zudem können Unfälle nicht vollständig und zuverlässig erfasst werden, die Dunkelziffer kann je nach Unfallart bis zu 85\% betragen \parencite[S.7-9]{Erke.1978}
