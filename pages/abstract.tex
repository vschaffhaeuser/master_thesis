\chapter{Kurzfassung}

%TODO: Abstract

Urbane Fahrsituationen sind weitaus komplexer als Fahrsituationen in Außerortsbereichen und führen dadurch zu einem erhöhten Unfallaufkommen. In Zukunft sollen automatisierte Fahrfunktionen auch im innerstädtischen Bereich zur Anwendung kommen, um die Anzahl der Verkehrsunfälle zu reduzieren. Unfalldatenanalysen tragen dazu bei, Fahrsituationen in Bezug auf ihre Sicherheitsrelevanz zu bewerten. Konflikte, die ein erhöhtes Risiko mit sich bringen, können so bei der Entwicklung automatisierter Systeme priorisiert werden.

Diese Arbeit beschäftigt sich mit der Analyse von Verkehrsunfalldaten eines Testgebiets im Münchner Norden. Es stehen Daten über einen Zeitraum von fünf Jahren (2012 bis 2016) für die Leopoldstraße, Ungererstraße und Schenkendorfstraße zur Verfügung. Zu Beginn werden, aufbauend auf einer umfangreichen Literaturrecherche, Hypothesen aufgestellt, die anhand der vorliegenden Verkehrsunfalldaten auf ihre Gültigkeit hin überprüft werden. Hierbei werden überwiegend die Unfallursachen zur Analyse herangezogen, da diese am meisten Auskunft über das Unfallgeschehen geben.

Anschließend wird eine Bewertungsmethode entwickelt, die es ermöglicht, die Unfälle innerhalb des Testgebiets bezüglich ihrer Sicherheitsrelevanz zu bewerten. Hierfür werden die Unfälle anhand der Unfallhäufigkeit und Unfallschwere verschiedenen Risiko-Kategorien zugeordnet. Wichtig ist, dass sowohl die Häufigkeit als auch die Schwere berücksichtigt werden, da Unfälle mit geringen Folgen, die dafür häufig auftreten, auch ein erhöhtes Risiko zur Folge haben.

Zur Bewertung der Unfälle werden die bei der Unfallaufnahme zugeordneten Unfalltypen verwendet. Da diese jedoch zu wenig Informationen über den genauen Unfallhergang liefern, werden sie mit Hilfe der vorliegenden Kurzsachverhalte (kurze Beschreibungen zum Unfallablauf) weiter unterteilt. Für diese Unterteilung werden die Feintypen des Gesamtverbandes der Deutschen Versicherungswirtschaft (GDV) herangezogen. Anschließend werden die Häufigkeit und die Schwere der jeweiligen Feintypen in ein Diagramm eingetragen und einer der zuvor definierten Risiko-Kategorie zugeordnet.

Für die Entwicklung automatisierter Systeme sind die Fahrsituationen relevant, die den Unfällen vorausgingen. Hierfür werden den Feintypen bestimmte Fahrsituationen innerhalb des Testgebiets zugeordnet. Die Zuordnung erfolgt mit Hilfe von Fahrsituation-Aufnahmen, die bei Testfahrten innerhalb des Untersuchungsgebiets entstanden sind, den vorliegenden Kurzsachverhalten und den erstellten Heatmaps. Die Heatmaps ermöglichen es, markante Punkte im Gebiet ausfindig zu machen.

Abschließend werden menschliche Fahrer mit automatisierten Systemen verglichen. Es werden bereits vorhandene Assistenzsysteme herangezogen, um besser bewerten zu können, welche Fahrsituationen mit zunehmender Automatisierung ein geringeres Risiko aufweisen bzw. in welchen sie an ihre Grenzen stoßen. 
