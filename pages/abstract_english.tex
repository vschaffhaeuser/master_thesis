\chapter{\abstractname}

Driving situations in urban areas are even more complex than those on highways or rural roads, leading to many more traffic accidents within urban areas. Future automated driving functions must be able to handle complex situations in cities to reduce the currently high number of accidents. Accident data analysis is an efficient option to evaluate the criticality of different driving situations. Subsequently the analysis results can be used to focus development of autonomous driving systems on those high critical situations.

This thesis deals with accident data over a five-year period (2012-2016) for a test region in the north of Munich. More precisely for three streets: Leopoldstraße, Ungererstraße and Schenkendorfstraße. After an extensive literature research hypotheses were fomulated and their validity should be proofed with the available accident data. The focus hereby lies on the causes of accidents because they provide most information about accident sequence.

Then a method to evaluate accidents in the test region regarding their safety issues is developed. Regarding frequency and severity of accidents, the accidents were assigned to different risk categories. It is important to consider frequency and severity because accidents with low consequences, which occur very often, can be assigned to a high-risk category just like accidents with high consequences but low frequency.

The type of accidents can be used for an initial evaluation. To get detailed information about the sequence of events short descriptions of each accident can be considered to divide the types of accidents into smaller segments. For this subdivision the fine accident types defined by \enquote{\acf{GDV}} are used. Subsequently the frequency and severity of fine accident types can be plotted to figure out their risk category.

The development of automated vehicles needs information about driving situations preceding traffic accidents. To show the relation between driving situations and accidents the fine accident types get linked with driving situations within the test area. Furthermore, heat maps can help to find out distinctive points with high accident rates.

Finally, this thesis compares human drivers with automated systems. Currently available functions on the market are taken into consideration. The aim of the comparison is to figure out, which driving situations have a lower or higher risk, if human drivers are replaced by computer systems.
