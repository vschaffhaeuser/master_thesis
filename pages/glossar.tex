% !TeX root = ../main.tex
% Add the above to each chapter to make compiling the PDF easier in some editors.

\phantomsection \addcontentsline{toc}{chapter}{Glossar}
\renewcommand\refname{Glossa} \chapter*{Glossar}
Definitionen entnommen aus \parencite{PolizeiprasidiumOberbeyernSud.2016}.\\
\\
\textit{Unfallart 0} - Unfall anderer Art\\
\\
\textit{Unfallart 1} - Zusammenstoß mit Fahrzeug, das anfährt, anhält, im ruhenden Verkehr steht.\\
\\
\textit{Unfallart 2} - Zusammenstoß mit Fahrzeug, das vorausfährt oder wartet.\\
\\
\textit{Unfallart 3} - Zusammenstoß mit Fahrzeug, das seitlich oder in gleicher Richtung fährt.\\
\\
\textit{Unfallart 4} - Zusammenstoß mit Fahrzeug, das entgegenkommt.\\
\\
\textit{Unfallart 5} - Zusammenstoß mit Fahrzeug, das einbiegt oder kreuzt.\\
\\
\textit{Unfallart 6} - Zusammenstoß zwischen Fahrzeug und Fußgänger.\\
\\
\textit{Unfallart 7} - Aufprall auf Hindernis auf der Fahrbahn.\\
\\
\textit{Unfallart 8} - Abkommen von der Fahrbahn nach rechts.\\
\\
\textit{Unfallart 9} - Abkommen von der Fahrbahn nach links.\\
\\
\textit{Unfalltyp 1} - Fahrunfall - Der Unfall wurde ausgelöst durch den Verlust der Kontrolle über das Fahrzeug, ohne Beitrag anderer Verkehrsteilnehmer. Infolge unkontrollierter Fahrzeugbewegungen kann es aber dann zur Kollision mit anderen Verkehrsteilnehmern gekommen sein.\\
\\
\textit{Unfalltyp 2} - Abbiege-Unfall - Der Unfall wurde ausgelöst durch den Konflikt zwischen einem Abbieger und einem aus gleicher oder entgegengesetzter Richtung kommenden Verkehrsteilnehmer an Kreuzungen, Einmündungen, Grundstücks- oder Parkplatzzufahrten.\\
\\
\textit{Unfalltyp 3} - Einbiegen/Kreuzen-Unfall - Der Unfall wurde ausgelöst durch den Konflikt zwischen einem einbiegenden oder kreuzenden und wartepflichtigen und einem vorfahrt-berechtigten Fahrzeug, an Kreuzungen, Einmündungen oder Ausfahrten von Grundstücken oder Parkplätzen.\\
\\
\textit{Unfalltyp 4} - Überschreiten-Unfall - Der Unfall wurde ausgelöst durch den Konflikt zwischen einem Fahrzeug und einem Fußgänger auf der Fahrbahn, sofern dieser nicht in Längsrichtung ging und sofern das Fahrzeug nicht abgebogen ist. Dies gilt auch, wenn der Fußgänger nicht erfasst wurde.\\
\\
\textit{Unfalltyp 5} - Unfall des ruhenden Verkehrs - Der Unfall wurde ausgelöst durch den Konflikt zwischen einem Fahrzeug des fließenden Verkehrs und einem Fahrzeug, das parkt, hält bzw. Fahrmanöver im Zusammenhang mit dem Parken/Halten durchführte.\\
\\
\textit{Unfalltyp 6} - Unfall im Längsverkehr - Der Unfall wurde durch den Konflikt zwischen Verkehrsteilnehmern, die sich in gleicher oder entgegengesetzter Richtung bewegten ausgelöst, sofern dieser Konflikt nicht einem anderen Unfalltyp entspricht.\\
\\
\textit{Unfalltyp 7} - Sonstiger Unfall - Unfall, der sich nicht den Typen 1-6 zuordnen lässt, z.B. Wenden, Rückwärtsfahren, Parker untereinander, Hindernis oder Tier auf der Fahrbahn, plötzlicher Fahrzeugschaden.\\
\\
\textit{Unfallursache 08} - Falschfahrt auf Straßen mit nach Fahrtrichtung getrennten Fahrbahnen (Stichwort \enquote{Falschfahrer}).\\
\\
\textit{Unfallursache 09} - Benutzung der Fahrbahn entgegen der vorgeschriebenen Fahrtrichtung in anderen Fällen (Stichwort \enquote{Einbahnstraße}).\\
\\
\textit{Unfallursache 10} - Verbotswidrige Benutzung der Fahrbahn oder anderer Straßenteile (z.B. Gehweg, Radweg).\\
\\
\textit{Unfallursache 11} - Verstoß gegen das Rechtsfahrgebot.\\
\\
\textit{Unfallursache 14} - Ungenügender Sicherheitsabstand.\\
\\
\textit{Unfallursache 15} - Starkes Bremsen des Vorausfahrenden ohne zwingenden Grund.\\
\\
\textit{Unfallursache 16} - Unzulässiges Rechtsüberholen.\\
\\
\textit{Unfallursache 17} - Überholen trotz Gegenverkehr.\\
\\
\textit{Unfallursache 18} - Überholen trotz unklarer Verkehrslage.\\
\\
\textit{Unfallursache 19} - Überholen trotz unzureichender Sicht.\\
\\
\textit{Unfallursache 20} - Überholen ohne Beachtung des nachfolgenden Verkehrs und/oder ohne rechtzeitige Ankündigung des Ausscherens.\\
\\
\textit{Unfallursache 21} - Fehler beim Wiedereinordnen nach rechts.\\
\\
\textit{Unfallursache 22} - Sonstige Fehler beim Überholen.\\
\\
\textit{Unfallursache 23} - Fehler beim Überholtwerden.\\
\\
\textit{Unfallursache 25} - Nichtbeachten des nachfolgenden Verkehrs beim Vorbeifahren an haltenden Fahrzeugen, Absperrungen oder Hindernissen und/oder ohne rechtzeitige deutliche Ankündigung des Ausscherens.\\
\\
\textit{Unfallursache 26} - Fehlerhaftes Wechseln des Fahrstreifens beim Nebeneinanderfahren oder Nichtbeachten des Reißverschlussverfahrens.\\
\\
\textit{Unfallursache 34} - Fehler beim Abbiegen nach rechts.\\
\\
\textit{Unfallursache 35} - Fehler beim Abbiegen nach links.\\
\\
\textit{Unfallursache 43} - Unzulässiges Halten oder Parken.\\
\\
\textit{Unfallursache 45} - Verkehrswidriges Verhalten beim Ein- oder Aussteigen, Be- und Entladen.\\
\\
\textit{Unfallursache 49} - Andere Fehler beim Fahrzeugführer.\\
\\
\textit{Unfallursache 60} - Falsches Verhalten der Fußgänger an Stellen, an denen der Fußgängerverkehr durch Polizeibeamte oder Lichtzeichen geregelt war.\\
\\
\textit{Unfallursache 61} - Falsches Verhalten der Fußgänger auf Fußgängerüberwegen ohne Verkehrsregelung durch Polizeibeamte oder Lichtzeichen.\\
\\
\textit{Unfallursache 62} - Falsches Verhalten der Fußgänger in der Nähe von Kreuzungen oder Einmündungen, Lichtzeichenanlagen oder Fußgängerüberwegen, bei dichtem Verkehr an anderer Stelle.\\
\\
\textit{Unfallursache 63} - Falsches Verhalten der Fußgänger durch plötzliches Hervortreten hinter Sichthindernissen.\\
\\
\textit{Unfallursache 64} - Falsches Verhalten der Fußgänger ohne auf den Fahrzeugverkehr zu achten.\\
\\
\textit{Unfallursache 65} - Falsches Verhalten der Fußgänger durch sonstiges falsches Verhalten.\\
\\
\textit{Unfallursache 66} - Nichtbenutzen des Gehwegs.\\
\\
\textit{Unfallursache 67} - Nichtbenutzen der vorgeschriebenen Straßenseite.\\
\\
\textit{Unfallursache 68} - Spielen auf/neben der Fahrbahn.\\
\\
\textit{Unfallursache 69} - Andere Fehler der Fußgänger.\\
\\
\textit{Unfallursache 70} - Verunreinigungen durch ausgegossenes Öl.\\
\\
\textit{Unfallursache 71} - Andere Verunreinigungen durch Straßenbenutzer.\\
\\
\textit{Unfallursache 72} - Schnee, Eis.\\
\\
\textit{Unfallursache 73} - Regen.\\
\\
\textit{Unfallursache 74} - Andere Einflüsse (Laub, etc.).\\
\\
\textit{Unfallursache 75} - Spurrillen im Zusammenhang mit Regen, Schnee oder Eis.\\
\\
\textit{Unfallursache 76} - Anderer Zustand der Straße.\\
\\
\textit{Unfallursache 77} - Nicht ordnungsgemäßer Zustand der Verkehrszeichen/-einrichtungen.\\
\\
\textit{Unfallursache 78} - Mangelhafte Beleuchtung der Straße.\\
\\
\textit{Unfallursache 79} - Mangelhafte Sicherung von Bahnübergängen.\\
\\
\textit{Unfallursache 80} - Nebel.\\
\\
\textit{Unfallursache 81} - Starker Regen, Hagel, Schnee.\\
\\
\textit{Unfallursache 82} - Blendende Sonne.\\
\\
\textit{Unfallursache 83} - Seitenwind.\\
\\
\textit{Unfallursache 84} - Unwetter oder sonstige Einflüsse.\\
\\
\textit{Unfallursache 85} - Nicht oder unzureichend gesicherte Arbeitsstelle auf der Fahrbahn.\\
\\
\textit{Unfallursache 86} - Wild auf der Fahrbahn.\\
\\
\textit{Unfallursache 87} - Anderes Tier auf der Fahrbahn.\\
\\
\textit{Unfallursache 88} - Sonstiges Hindernis auf der Fahrbahn.\\
\\
\textit{Unfallursache 89} - Sonstige äußere Ursache.\\
\\
\textit{Unfallursache 90} - Schädigung der Fahrbahnoberfläche.\\