% !TeX root = ../main.tex
% Add the above to each chapter to make compiling the PDF easier in some editors.

\phantomsection \addcontentsline{toc}{chapter}{Abkürzungsverzeichnis}
\renewcommand\refname{Abkürzungsverzeichnis} \chapter*{Abkürzungsverzeichnis}
\begin{acronym}[NMWC] % längste Abkürzung steht in eckigen Klammern
	\setlength{\itemsep}{-\parsep} % geringerer Zeilenabstand
	\acro{ABS}{Antiblockiersystem}
	\acro{ASIL}{Automotive Safety Integrity Level}
	\acroplural{ASIL}[ASILs]{Automotive Safety Integrity Levels}
	\acro{BArt}{Beteiligungsart}	
	\acro{BASt}{Bundesanstalt für Straßenwesen}
	\acro{ESP}{Elektronisches Stabilitätsprogramm}	
	\acro{FAS}{Fahrerassistenzsystem}
	\acro{FAT}{Forschungsvereinigung Automobiltechnik}	
	\acro{GDV}{Gesamtverband der Deutschen Versicherungswirtschaft}	
	\acro{GIDAS}{German in Depth Accident Study}
	\acro{IO}{Infrastructure-Only}	
	\acro{K}{Kleinunfall}
	\acro{Kfz}{Kraftfahrzeug}
	\acro{kvl}{keine Verletzung}
	\acroplural{Kfz}[Kfzs]{Kraftfahrzeugen}
	\acro{LSA}{Lichtsignalanlage}
	\acro{lvl}{leichtverletzt}
	\acro{ÖPNV}{öffentlicher Personennahverkehr}
	\acro{P}{Personenschaden}
	\acro{QM}{Qualitätsmanagement}	
	\acro{S}{schwerwiegender Sachschaden im engeren Sinne}
	\acro{svl}{schwerverletzt}
	\acro{tot}{getötet}
	\acro{Urs}{Ursache}
	\acro{V2I}{Vehicle-to-Infrastructure}
	\acro{V2V}{Vehicle-to-Vehicle}
\end{acronym}