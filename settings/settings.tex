\addbibresource{biblatex-examples.bib}
\renewcommand*{\nameyeardelim}{\addcomma\space}
\renewcommand*{\mkbibnamelast}[1]{\textsc{#1}}

\DefineBibliographyStrings{ngerman}{andothers={et al\adddot}}

\DeclareNameAlias{author}{last-first}

\NewBibliographyString{langslovene}
\NewBibliographyString{fromslovene}

\bibliography{bibliography/BibTeX}

\setkomafont{disposition}{\normalfont\bfseries} % use serif font for headings
\linespread{1.05} % adjust line spread for mathpazo font

% Settings for glossaries TODO: remove the following block if glossary not needed
\renewcommand{\glsnamefont}[1]{\normalfont\bfseries #1} % use serif font for glossary entry titles
\makeglossaries{}

% Settings for pgfplots
\pgfplotsset{compat=1.9} % TODO: adjust to your installed version
\pgfplotsset{
  % For available color names, see http://www.latextemplates.com/svgnames-colors
  cycle list={CornflowerBlue\\Dandelion\\ForestGreen\\BrickRed\\},
}

% Settings for lstlistings
\lstset{%
  basicstyle=\ttfamily,
  columns=fullflexible,
  autogobble,
  keywordstyle=\bfseries\color{MediumBlue},
  stringstyle=\color{DarkGreen}
}


%separaates Anhangsverzeichnis
\makeatletter% --> De-TeX-FAQ
% Weitergabe des folgenden Codes oder Modifikationen davon nur unter Nennung
% der Originalquelle: <http://www.komascript.de/comment/1073#comment-1073>,
% gestattet.
% Leistungsfähigere Lösung unter <http://www.komascript.de/comment/3447#comment-3447>.
\newcommand*{\maintoc}{% Hauptinhaltsverzeichnis
	\begingroup
	\@fileswfalse% kein neues Verzeichnis öffnen
	\renewcommand*{\appendixattoc}{% Trennanweisung im Inhaltsverzeichnis
		\value{tocdepth}=-10000 % lokal tocdepth auf sehr kleinen Wert setzen
	}%
	\tableofcontents% Verzeichnis ausgeben
	\endgroup
}
\newcommand*{\appendixtoc}{% Anhangsinhaltsverzeichnis
	\begingroup
	\edef\@alltocdepth{\the\value{tocdepth}}% tocdepth merken
	\setcounter{tocdepth}{-10000}% Keine Verzeichniseinträge
	\renewcommand*{\contentsname}{% Verzeichnisname ändern
		Anlagenverzeichnis}%
	\renewcommand*{\appendixattoc}{% Trennanweisung im Inhaltsverzeichnis
		\setcounter{tocdepth}{\@alltocdepth}% tocdepth wiederherstellen
	}%
	\tableofcontents% Verzeichnis ausgeben
	\setcounter{tocdepth}{\@alltocdepth}% tocdepth wiederherstellen
	\endgroup
}
\newcommand*{\appendixattoc}{% Trennanweisung im Inhaltsverzeichnis
}
\g@addto@macro\appendix{% \appendix erweitern
	\if@openright\cleardoublepage\else\clearpage\fi% Neue Seite
	\addcontentsline{toc}{chapter}{\appendixname}% Eintrag ins Hauptverzeichnis
	\addtocontents{toc}{\protect\appendixattoc}% Trennanweisung in die toc-Datei
}
\makeatother
